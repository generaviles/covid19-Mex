\documentclass[]{article}
\usepackage{lmodern}
\usepackage{amssymb,amsmath}
\usepackage{ifxetex,ifluatex}
\usepackage{fixltx2e} % provides \textsubscript
\ifnum 0\ifxetex 1\fi\ifluatex 1\fi=0 % if pdftex
  \usepackage[T1]{fontenc}
  \usepackage[utf8]{inputenc}
\else % if luatex or xelatex
  \ifxetex
    \usepackage{mathspec}
  \else
    \usepackage{fontspec}
  \fi
  \defaultfontfeatures{Ligatures=TeX,Scale=MatchLowercase}
\fi
% use upquote if available, for straight quotes in verbatim environments
\IfFileExists{upquote.sty}{\usepackage{upquote}}{}
% use microtype if available
\IfFileExists{microtype.sty}{%
\usepackage[]{microtype}
\UseMicrotypeSet[protrusion]{basicmath} % disable protrusion for tt fonts
}{}
\PassOptionsToPackage{hyphens}{url} % url is loaded by hyperref
\usepackage[unicode=true]{hyperref}
\hypersetup{
            pdftitle={Recomendaciones Técnicas para la Iglesia Adventista del Séptimo Día},
            pdfborder={0 0 0},
            breaklinks=true}
\urlstyle{same}  % don't use monospace font for urls
\usepackage[margin=1in]{geometry}
\usepackage{longtable,booktabs}
% Fix footnotes in tables (requires footnote package)
\IfFileExists{footnote.sty}{\usepackage{footnote}\makesavenoteenv{long table}}{}
\usepackage{graphicx,grffile}
\makeatletter
\def\maxwidth{\ifdim\Gin@nat@width>\linewidth\linewidth\else\Gin@nat@width\fi}
\def\maxheight{\ifdim\Gin@nat@height>\textheight\textheight\else\Gin@nat@height\fi}
\makeatother
% Scale images if necessary, so that they will not overflow the page
% margins by default, and it is still possible to overwrite the defaults
% using explicit options in \includegraphics[width, height, ...]{}
\setkeys{Gin}{width=\maxwidth,height=\maxheight,keepaspectratio}
\IfFileExists{parskip.sty}{%
\usepackage{parskip}
}{% else
\setlength{\parindent}{0pt}
\setlength{\parskip}{6pt plus 2pt minus 1pt}
}
\setlength{\emergencystretch}{3em}  % prevent overfull lines
\providecommand{\tightlist}{%
  \setlength{\itemsep}{0pt}\setlength{\parskip}{0pt}}
\setcounter{secnumdepth}{0}
% Redefines (sub)paragraphs to behave more like sections
\ifx\paragraph\undefined\else
\let\oldparagraph\paragraph
\renewcommand{\paragraph}[1]{\oldparagraph{#1}\mbox{}}
\fi
\ifx\subparagraph\undefined\else
\let\oldsubparagraph\subparagraph
\renewcommand{\subparagraph}[1]{\oldsubparagraph{#1}\mbox{}}
\fi

% set default figure placement to htbp
\makeatletter
\def\fps@figure{htbp}
\makeatother

\usepackage{etoolbox}
\makeatletter
\providecommand{\subtitle}[1]{% add subtitle to \maketitle
  \apptocmd{\@title}{\par {\large #1 \par}}{}{}
}
\makeatother

\title{Recomendaciones Técnicas para la Iglesia Adventista del Séptimo Día}
\providecommand{\subtitle}[1]{}
\subtitle{Asociación de Baja California}
\author{}
\date{\vspace{-2.5em}18 de Mayo 2020}

\begin{document}
\maketitle

\subsection{Generalidades}\label{generalidades}

La Secretaría de Salud Federal, a través de la Subsecretaría de
Prevención y Promoción de la Salud, ha publicado lineamientos
específicos para la reactivación de espacios públicos cerrados como
parte de la Fase 3 Epidemiológica de la contingencia generada por
COVID-19 (1). En éste documento se detallan las acciones recomendadas
para espacios públicos (a lo que los espacios de culto pertenecen). A
continuación se presenta un resumen de las recomendaciones pertinentes
para los centros de culto de la Iglesia Adventista del Séptimo Día en la
Asociación de Baja California.

El objetivo principal en las fases de desescalamiento es la
\textbf{disminución de la velocidad de transmisión y contagio de
COVID-19 en la población}, a través de estrategias de mitigación y
prevención, \textbf{incrementando la protección a grupos en mayor
riesgo} ((2)) como lo son:

\begin{itemize}
\tightlist
\item
  mujeres embarazadas,
\item
  mayores de 60 años,
\item
  personas con enfermedades crónicas:

  \begin{itemize}
  \tightlist
  \item
    cardiopulmonares,
  \item
    inmunodeprimidos,
  \item
    diabettes mellitus,
  \item
    obesos mórbidos,
  \item
    entre otros.
  \end{itemize}
\end{itemize}

\subsection{Medidas de mitigación no
médicas}\label{medidas-de-mitigaciuxf3n-no-muxe9dicas}

Las acciones de mitigación no médicas recomendadas por la autoridad
federal en salud se pueden dividir en tres subconjuntos:

\begin{enumerate}
\def\labelenumi{\arabic{enumi}.}
\tightlist
\item
  Medidas de higiene.
\item
  Filtro de supervisión.
\item
  Sana distancia.
\end{enumerate}

\subsubsection{Medidas de higiene}\label{medidas-de-higiene}

\begin{itemize}
\tightlist
\item
  Lavado de manos (15-20 veces diario).
\item
  Uso de tapabocas en todo momento en espacio público cuando semáforo de
  rojo a amarillo.
\item
  Resguardo en casa si hay síntomatología respiratoria.
\item
  Mantener higiene adecuada del espacio cerrado.

  \begin{itemize}
  \tightlist
  \item
    Limpieza completa de superficies antes y después de ser utilizadas.
  \item
    Ventilación natural de todas las áreas cerradas.
  \item
    Se puede utilizar agua y jabón comercial para limpieza general de
    pisos y superficies grandes.
  \item
    Se puede utilizar solución de hipoclorito de sodio (cloro comercial)
    diluida después del uso del jabón y agua.
  \end{itemize}
\item
  Evitar contacto físico de cualquier tipo: estrechar manos, abrazos,
  besos.
\item
  Lavado y planchado frecuente de telas, cortinas, manteles, étc.
  Utilizados en espacio público.
\end{itemize}

\subsubsection{Filtro de supervisión}\label{filtro-de-supervisiuxf3n}

El objetivo principal es disminuir la probabilidad de que las personas
contenidas en el espacio cerrado presenten un riesgo potencial de
contagio para el resto de los asistentes. La propuesta federal es que el
módulo de supervisión incluya:

\begin{itemize}
\tightlist
\item
  Aplicación de un cuestionario breve sobre estado de salud,
\item
  aplicación de gel antibacterial,
\item
  información sobre estrategias de mitigación de COVID-19
\item
  indicación de unidades de salud más cercana.
\end{itemize}

\subsubsection{Sana distancia}\label{sana-distancia}

Dependiendo el escenario en el que se encuentre cada uno de los
municipios de Baja California, la sana distancia puede ir de 1.80 a 2.25
metros entre persona y persona como mínimo. A continuación se reproduce
la tabla proporcionada en los lineamientos oficiales:

\begin{longtable}[]{@{}cccc@{}}
\toprule
Escenario & Sana Dist Mínima & Dist Adicional & Dist
Total\tabularnewline
\midrule
\endhead
1 & 1.5m & 0.30m & 1.80m\tabularnewline
2 & 1.5m & 0.45m & 1.95m\tabularnewline
3 & 1.5m & 0.75m & 2.25m\tabularnewline
\bottomrule
\end{longtable}

Éstas indicaciones permite realizar un cálculo del máximo de personas
que pueden estar presentes al mismo tiempo en cada iglesia, dadas las
dimensiones de cada templo. Cabe destacar que a la fecha de generación
de éste documento \textbf{todo el estado de Baja California está
catalogado en el Escenario 3}.

\subsection{Reflexiones y puntos
claves}\label{reflexiones-y-puntos-claves}

El desescalamiento en el estado de Baja California es una realidad que,
a la fecha de generación de éste documento, se percibe distante ya que
éste estado sigue clasificado en el escenario epidemiológico número 3
por el Gobierno Federal. No obstante todo ejercicio de planeación y
preparación para éstos eventos es un acto de responsabilidad y
profesionalismo por parte de los dirigentes de la IASD en éste Estado.
Buscando brevedad en el documento, a continuación se resume puntualmente
algunas conclusiones:

\begin{enumerate}
\def\labelenumi{\arabic{enumi}.}
\tightlist
\item
  Las iglesias en Baja California, sus dirigentes y laicos deben entrar
  en una fase de preparación intensa para poder alinearse a las
  estrategias de higiene, filtros de supervisión y sana distancia
  propuestas por el gobierno para así proteger a la hermandad.
\item
  Es esperable que un porcentaje significativo de la hermandad
  (probablemente cercano al 50\%) sea clasificada como población en
  riesgo y no puedan asistir a los templos hasta que el semáforo
  epidemiológico cambie a verde.
\item
  Los cultos en línea parecen ser herramientas que podrían ser de gran
  apoyo durante el proceso de desescalamiento, aún cuando porciones de
  la población inicien a asistir en persona a los templos.
\item
  Las reuniones \emph{masivas} (\textgreater{}50 personas manteniendo
  sana distancia) no se presentan como viables, por lo menos en el
  futuro inmediato y a mediano plazo.
\item
  Nos enfrentamos a un cambio de paradigma en la vida eclesiástica,
  parece ser que lo más racional por hacer ahora no es anhelar el volver
  a prácticas pasadas sino reinventarse y transformar nuestra
  experiencia religiosa de tal forma que nos permita disminuir las
  probabilidades de propagación de COVID-19 al mismo tiempo que nos
  mantenga viviendo una vibrante vida adventista en comunidad.
\end{enumerate}

\emph{Documento compilado por Gener José Avilés Rodríguez, Lic.Med, MSc,
DrSc(c), para la Asociación de Baja California de los Adventistas del
Séptimo Día.}

\subsection*{Referencias}\label{referencias}
\addcontentsline{toc}{subsection}{Referencias}

\hypertarget{refs}{}
\hypertarget{ref-Lineamie24:online}{}
1. Subsecretaría de Prevención y Promoción de la Salud M.
Lineamiento\_Espacio\_Cerrado\_27Mar2020\_1830.
\url{https://coronavirus.gob.mx/wp-content/uploads/2020/03/Lineamiento_Espacio_Cerrado_27032020.pdf};
2020.

\hypertarget{ref-DatosAbi46:online}{}
2. Subsecretaría de Prevención y Promoción de la Salud M. Datos Abiertos
- Dirección General de Epidemiología \textbar{} Secretaría de Salud
\textbar{} Gobierno \textbar{} gob.mx.
\url{https://www.gob.mx/salud/documentos/datos-abiertos-152127};
17n.~Chr.

\end{document}
